\documentclass[a4paper, danish, 10pt, final]{article}
%%%%%%%%%%%%%%%%%%%%%%%%%%%%%%%%%%%%%%%%%%%%%%%%%%%%%%%%%%%%%%%%%%
% Pakker
%%%%%%%%%%%%%%%%%%%%%%%%%%%%%%%%%%%%%%%%%%%%%%%%%%%%%%%%%%%%%%%%%%
\usepackage{a4wide}
\usepackage[utf8]{inputenc}
\usepackage[T1]{fontenc}
\usepackage[danish]{babel}
\usepackage{verbatim}
\usepackage{amsfonts}
\usepackage{amsmath}
\usepackage{amssymb}
\usepackage{mathrsfs}
\usepackage[mathcal]{euscript}
\usepackage{listings}
\usepackage{charter}
\usepackage{epsfig}
\usepackage{graphicx}
\usepackage{url}
\usepackage[numbers]{natbib}
\usepackage{hyperref}

\hypersetup{
colorlinks,%
citecolor=red,%
filecolor=black,%
linkcolor=black,%
urlcolor=blue,%
bookmarksopen=false,
pdftitle={Ønskeliste},
pdfauthor={Ulrik Bonde}
}

%%%%%%%%%%%%%%%%%%%%%%%%%%%%%%%%%%%%%%%%%%%%%%%%%%%%%%%%%%%%%%%%%%%%%%%%%%%%%%%%%%%%%
% Indstillinger
%%%%%%%%%%%%%%%%%%%%%%%%%%%%%%%%%%%%%%%%%%%%%%%%%%%%%%%%%%%%%%%%%%%%%%%%%%%%%%%%%%%%%
\parindent=5pt
\parskip=8pt plus 2pt minus 4pt

%%%%%%%%%%%%%%%%%%%%%%%%%%%%%%%%%%%%%%%%%%%%%%%%%%%%%%%%%%%%%%%%%%%%%%%%%%%%%%%%%%%%%
% Titel, forfatter og dato
%%%%%%%%%%%%%%%%%%%%%%%%%%%%%%%%%%%%%%%%%%%%%%%%%%%%%%%%%%%%%%%%%%%%%%%%%%%%%%%%%%%%%
\title{Ønskeliste}
\author{Ulrik Bonde -- ulrikbonde@gmail.com}
\date{\today}

%%%%%%%%%%%%%%%%%%%%%%%%%%%%%%%%%%%%%%%%%%%%%%%%%%%%%%%%%%%%%%%%%%%%%%%%%%%%%%%%%%%%%
% Indhold
%%%%%%%%%%%%%%%%%%%%%%%%%%%%%%%%%%%%%%%%%%%%%%%%%%%%%%%%%%%%%%%%%%%%%%%%%%%%%%%%%%%%%
\begin{document}
\maketitle

\subsection*{Bøger}

Rækkefølgen er nogenlunde prioriteret.
\begin{itemize}
    \item Programming Python \citep{lutz2011progpython}
    \item JavaScript and jQuery: The Missing Manual, 2nd Edition \cite{javascriptjquery}
    \item HTML and XHTML Pocket Reference, 4th Edition \cite{htmlxhtml}
    \item The Manga Guide to Calculus \citep{kojima2009calc}
    \item sed and awk Pocket Reference \citep{robbins200206}
    \item grep Pocket Reference \citep{bambenekKlus200901}
    \item Python Pocket Reference \citep{lutz2009python}
\end{itemize}
De resterende er knapt så aktuelle...
\begin{itemize}
    \item Programming Erlang \citep{armstrong2007pe}
    \item GNU Emacs Pocket Reference \citep{cameron199811}
    \item Real World Haskell \citep{osullivanGoerzenStewart200812}
    \item Learning the vi and Vim Editors \citep{robbinsHannahLamb200807}
\end{itemize}

\subsection*{Rubiks Cubes og andet nørd udstyr}

Enhver kan bruges, men kig gerne efter \emph{Shenshou V.3 $4\times4\times4$} på
\cite{gamesweb} eller \emph{Dayan+Mf8 4x4} hos \cite{cubikon}. Find noget
skægt, måske kæmpestore kuber. Jeg kunne også godt bruge en ny type A V (læses
        A-5, se hos \cite{gamesweb}) og nogle nye klistermærker fra
\cite{cubesmith}. \emph{QJ 4x4} virker også interessant. nanodots fra
\cite{gamesweb} er også sjove.

\subsection*{Musik/Film}

Gå altid efter limited/special editions.
\begin{itemize}
    \item Devin Townsend: Contain Us (Box Set Ltd. Edition 6CD + 2DVD)
    \item Pain of Salvation: Road Salt Two (Ltd. Edition)
    \item Dream Theater: A Dramatic Turn of Events (CD + DVD)
    \item Megadeth: Th1rt3en
    \item Kashmir: Katalgoue (2CD)
    \item Harry Potter Box Set (8DVD)
\end{itemize}

\subsection*{Praktisk}
\begin{itemize}
    \item Vinterjakke.
    \item Diverse skruetrækkere.
    \item Samsung Galaxy S II (Android telefon).
    \item Samsung Galaxy Tab (Android tablet).
    \item Ny monitor til computer (Samsung, den eksisterende er super-skod).
    \item Stort batteri til IBM Thinkpad X41 (eller ny maskine).
    \item Et akvarie med kæmpeblæksprutter.
\end{itemize}


\bibliographystyle{unsrtnat}
\bibliography{litteratur}
%\addcontentsline{toc}{chapter}{Litteratur}

\end{document}
